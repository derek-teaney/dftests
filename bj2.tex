\documentclass[12pt,aps,preprint,nofootinbib,noshowkeys,noshowpacs,longbibliography,superscriptaddress,tightenlines]{revtex4-2}
\usepackage{amsmath,amsfonts,amsthm,amssymb}
%\usepackage{showkeys}
\usepackage{graphics,graphicx}
\usepackage[inline]{enumitem}
\usepackage{mathrsfs}
\usepackage{bm}
\usepackage{color}

\definecolor{darkblue}{RGB}{0,0,196}
\definecolor{darkgreen}{RGB}{0,120,0}
%\usepackage{cancel}
% \usepackage{caption}
%\usepackage{bbold}
%\usepackage{subfigure}
%\usepackage{stix}
%\usepackage{multirow}
%\usepackage{longtable}
%\usepackage{color}
%\usepackage[normalem]{ulem}
%\usepackage{hyperref}
%\usepackage{authblk}
%\usepackage{bigints}
%\usepackage{xparse}
%\usepackage{physics}
%\usepackage{verbatim}
%\usepackage{minibox}
%\usepackage{comment}
%\usepackage{appendix}
%%%\usepackage[inline]{showlabels}
%\usepackage{scalerel}
%\usepackage{marginnote}
%\usepackage{graphicx}
%\usepackage[nice]{nicefrac}
%\usepackage{soul}
\newcommand{\mathcolorbox}[2]{\colorbox{#1}{$\displaystyle #2$}}
\newcommand{\hlfancy}[2]{\sethlcolor{#1}\hl{#2}}
% -- highlighting
\usepackage{stackengine}
\renewcommand\useanchorwidth{T}
\newcommand\oU[1]{\ensurestackMath{\stackon[1pt]{#1}{\mkern2mu\bullet}}}
\newcommand\oX[1]{\ensurestackMath{\stackon[1pt]{#1}{\mkern2mu\star}}}
\newcommand\oY[1]{\ensurestackMath{\stackon[1pt]{#1}{\mkern2mu\smwhitestar}}}
\newcommand\oZ[1]{\ensurestackMath{\stackon[1pt]{#1}{\mkern2mu\scaleto{\circ}{3pt}}}}
\renewcommand\S{\mathcal S}
\newcommand\Sstress{{\mathcal S}}
\renewcommand\P{\mathcal P}
\newcommand\N{\mathcal N}
\def\half{\tfrac{1}{2}}
\def\quarter{\tfrac{1}{4}}
\def\LF{{\scriptscriptstyle LF}}
\newcommand{\ubeta}{{\underline{\smash{\beta}}}}
\newcommand{\uf}{{\underline{{f}}}}
\newcommand{\uu}{{\underline{u}}}
\def\pp{{\mathfrak p}}
\def\three{{}^{(3)}}
\def\bb{{(b)}}
\def\x{{\bm x}}
\def\p{{\bm p}}
\def\edense{{\mathcal{E}}}
\def\pdense{M}
\def\dpdense{\delta\kern-0.08em M}
\def\dbeta{\delta\kern-0.08em \beta}
\def\CChi{ {\mathcal \chi}}
\def\HP{\hphantom{\alpha}} % horizontal
\def\llangle{\left\langle}
\def\rrangle{\right\rangle}
\newcommand{\nn}{\nonumber}
\newcommand{\rl}{\mathrm{rel}}
\newcommand{\tbj}{\tau}
\newcommand{\vecxp}{\vec{x}_\perp}
\def\muh{\hat\mu}
\def\vh{\hat v}
\def\noisekernel{{\kappa}}
\def\Eq#1{eq.~(\ref{#1})}
\def\Eqs#1{eqs.~(\ref{#1})}
\def\eq#1{(\ref{#1})}
\def\app#1{App.~\ref{#1}}
\def\Fig#1{Fig.~\ref{#1}}
\def\Figs#1{Figs.~\ref{#1}}
\def\Sect#1{Sect.~\ref{#1}}
\def\Ref#1{Ref.~\cite{#1}}


% equation environments  
\def\P{\mathcal P}
\def\Kd{{K^\tau}}
\def\beq{\begin{equation}}
\def\eeq{\end{equation}}
\def\st{\begin{equation}}
\def\stp{\end{equation}}
\def\ba{\begin{eqnarray}}
\def\ea{\end{eqnarray}}   
\def\DD{{\mathcal L}}
\def\vDD{{\mathcal L}}
\newcommand\vbeta{{\vec{\beta}}}
\newcommand\vcovD{D}
\newcommand\Chi{\mathcal X}
\newcommand\covD{\nabla}
\newcommand{\ve}[2]{e_{\;\,#1}^{#2}}
\newcommand{\de}[2]{e_{\;\,#2}^{#1}}
\newcommand{\vp}{{\vec{p}\,}}
\newcommand{\petsc}{{\tt PETsc}}

\def\dd{{\rm d}}
\def\lps{{N}}
\def\sp{\phantom{j}}
\def\spm{\phantom{\mu}}
\def\sft{\vec{N}}
\def\tideal{{\mathcal T}}
\def\tvisc{\Pi}
\def\btvisc{\bar{\Pi}}
\newcommand\ph{\phantom{\nu}}

\newcommand\V{V_0}
\newcommand\dtau{{{\Delta}\tau}}
\newcommand\dt{{{\Delta}t}}

%\usepackage{showkeys}
\def\bb1{{(1)}}
\def\bbt{{(2)}}
\def\bbth{{(3)}}

\newcommand{\AM}[1]{{\color{green}AM: #1}}

\newcommand{\cmtJF}[1]{{\color{blue}JF: #1}}
\newcommand{\rs}[1]{{\color{red}RS: #1}}
\newcommand{\cmtMS}[1]{{\color{cyan}MS: #1}}

\newcommand{\gammathree}{{}^{(3)}\Gamma}
\usepackage[colorlinks=true,linktocpage=true,linkcolor=darkblue,citecolor=red,urlcolor=darkblue]{hyperref}
\usepackage{cleveref}

\renewcommand{\baselinestretch}{1.05}

\begin{document}
\preprint{}
 
\title{Relativistic Viscous Hydrodynamics in the Density Frame: Numerical Tests and Comparisons}

\section{Introduction} 


\section{The Density Frame for 2+1D Bjorken Flow}
\label{sec:densityframe}

The numerical method presented in here is based on the Density Frame formulation of relativistic viscous hydrodynamics, which is described in detail in our prior numerical work~\cite{Numerics1D}.   Here we will 
focus on the extension to the Bjorken setup, initially describing 
the numerical method in 1+1D for simplicity. With the right notation and concepts the extension of the numerical method to 2+1D is straightforward. 
Relative to our previous work, the main new ingredient is to maximally use notion of the dissipative function $R$~\cite{}, which is the positive definite quadratic form that appears in the entropy production.  We will see that 
it is possible to write down the dissipative function in a discrete form, and that the viscous strains can be implemented by taking derivatives of this function at the discrete level.  This approach is very effective and leads to a delightfully stable numerical method.

\subsection{Preliminaries and the dissipative function}

Let's consider a special case of Bjorken Flow with a non-zero flow velocity and spatial dependence only 
in the $x$ direction.
The metric in Bjorken coordinates is 
\st
   ds^2 = - \dd\tau^2 + (\dd x^2 + \dd y^2 + \tau^2 \dd\eta^2) \, ,
\stp
and the non-zero Christoffel symbols are $\Gamma^\tau_{\eta\eta} = \tau$ and
$\Gamma^\eta_{\tau\eta} = 1/\tau$.  

The normal vector for the foliation is simply $n^{\mu}=(1, 0, 0, 0)$
and the spatial metric is $h_{ij} = {\rm diag}(1, 1, \tau^2)$ where roman indices $i,j,k$ run over the spatial coordinates $x, y, \eta$. 
The volume element
of the spatial metric is $\sqrt{h} = \tau$. 
In two dimension,  we will use $a,b,c$ for the transverse directions  $x,y$.
The foliation provided by the Bjorken
coordinates has extrinsic curvature $K_{ij} =  -n_{\mu }\Gamma^{\mu}_{ij} $,
which has only one non-zero component $K_{\eta\eta} = -\Gamma^{\tau}_{\eta\eta}$.  To higlight the role of the extrinsic curvature we will write $\Gamma^{\tau}_{\eta\eta}$ instead of simply $\tau$ in the equations of motion.


The evolution variables in hydrodynamics are the energy and momentum densities 
\st
( T^{\tau\tau},  T^{\tau x} ) \equiv (\edense , \pdense^x)\,,
\stp
and the equations of energy-momentum conservation reads
  \begin{subequations}
  \begin{align}
  \partial_\tau (\tau \edense) + \partial_x  (\tau \pdense^x)  + \tau \Gamma^{\tau}_{\eta\eta} T^{\eta\eta}=&  0 \, , 
   \label{eq:equation}   \\
  \partial_\tau  (\tau \pdense^x) + \partial_x (\tau T^{xx}) =&0 \,.  
  \label{eq:pequation}
\end{align} 
  \end{subequations}
These equations are extended to two dimensions in an obvious way by including the $y$ direction and the corresponding momentum density $\pdense^y$. The explicit factors of $\tau$ represent the volume element, $\sqrt{h}$. 


The conservation equations are closed by specifying a
consitutive relation for the spatial stress tensor $T^{ij}$ in terms of
$(\edense, \pdense^x)$ and their spatial derivatives.
In ideal hydrodynamics this is  conveneintly done by introducing intermediate parameters $\beta_{\mu}\equiv \beta u_{\mu}$ (the inverse temperature and flow velocity), which are defined by algebraic relations to the  conserved charge densities on the spatial slice:
\begin{subequations}
  \label{eq:densityframedef}
\begin{align}
  \edense =& \tideal^{\tau\tau}(\beta) \, , \\
  \pdense^x =& \tideal^{\tau x}(\beta)  \,  , 
\end{align}
\end{subequations}
Here the ideal stress tensor written here takes the  
the functional form
\st
\tideal^{\mu\nu}(\beta) \equiv \left(e(\beta) + p(\beta)\right)  u^{\mu} u^{\nu} + p(\beta) g^{\mu\nu} \, . 
\stp
This equation  is to be understood as follows: $\beta_{\mu}$ is determined from the energy and momentum densities as specified by \Eq{eq:densityframedef},  and subsequently 
subsequently $\beta_{\mu}$ is used to specify the spatial stress 
\begin{align}
   \tideal^{ij}(\beta) =& \left(e(\beta) + p(\beta)\right)  u^{i} u^{j} + p(\beta) g^{ij} \, .
\end{align}
It is important that $T^{ij}$ is ultimately a function of $(\edense, \pdense^x)$.

In viscous hydrodynamics the spatial stress tensor  receives derivative corrections
\st
 T^{ij} = \tideal^{ij}(\beta) + \Pi^{ij}
\stp
In the Density Frame the algebraic relations in \Eq{eq:densityframedef} define $\beta_{\mu}(x)$ to all orders in the derivative expansion.
% , but corrections arrise order by order in gradients.
% The viscous stress tensor  is
% \st
%  T^{ij} = \tideal^{ij} + \Pi^{ij}
% \stp
The viscous strains in Bjorken coordinates is proportional to the spatial covariant derivatives of $\beta_{\mu}$:
\st
\label{eq:covderiv}
\Pi^{ij} = -T\kappa^{ijmn} \nabla_{m} \beta_n \equiv - T\kappa^{ijmn} \left( \partial_m \beta_n - \beta_{\tau} \Gamma^{\tau}_{mn} \right)
\stp
A general expression for the tensor $\kappa^{ijmn}(v)$ is given in our companion paper and depends on the shear and bulk viscosities, the flow flow velocity of the fluid $v^i = \beta^i/\beta^{\tau}$, and the speed of sound~\cite{TheoryPaper}.  In  $1+1$ dimensions with only $x$ dependence,  the symmetries of the problem imply that the only relevant components of 
  $T\kappa^{ijmn}$ are 
% \st
% \left\{ \kappa^{xxxx}, \tau^2 \kappa^{xx\eta\eta} , \tau^4 \kappa^{\eta\eta \eta\eta} \right\} =  \eta \left\{\frac{4 }{3 \gamma^4 \left(1- c^2
%     v^2\right)^2},-\frac{2  \left(1 -3
%     c^2 v^2\right)}{3  \gamma^2 \left(1- c^2
%     v^2\right)^2},
%     \frac{4 \left(3 c^4 v^4-3 c^2
%     v^2+1\right)}{3 \left(1 - c^2 v^2\right)^2}\right\}
% \stp
% as well as some others which do not contribute to $T^{\eta\eta}$ or the net force $\partial_j \Pi^{ij}$. 
% $T^{xx}$ and $T^{\eta\eta}$ receive viscous corrections of order $\partial_x\beta_x$ and $\tau^2 K_{\eta\eta}$. 
% We have the $xx$ component
% \st
% \Pi^{xx} \equiv  - T\noisekernel^{xxxx} \,  \partial_{(x}\beta_{x)}  - T \tau^2 \noisekernel^{xx\eta\eta} \; \frac{\beta^\tau }{\tau}  \, , 
% \stp
% and the $\eta\eta$ component
% \st
% \tau^2 \Pi^{\eta\eta} \equiv  - T \tau^2\noisekernel^{\eta\eta x x} \, \partial_{(x}\beta_{x)}  - T  \tau^4 \noisekernel^{\eta\eta\eta\eta} \; \frac{\beta^\tau}{\tau} \, , 
% \stp

% We   simplify  terms in these expressions 
\begin{align}
\label{eq:noisekernelintro}
\noisekernel^{xxxx} =&     \frac{1}{(1 -c_s^2 v^2)^2 \gamma^4}   \,  \left( \frac{4\eta}{3}  + \zeta \right)    \,. \\
\tau^2\noisekernel^{xx\eta\eta} =&   \frac{1}{(1 - c_s^2 v^2)^2 \gamma^2}  \left(   -  \frac{2\eta}{3} (1 - 3 c_s^2 v^2)  +  \zeta \right) \,. \\
\tau^4\noisekernel^{\eta\eta\eta\eta} =&   \frac{1}{(1- c_s^2 v^2)^2} \,   \left( \eta (1 - c_s^2 v^2)^2 + \frac{\eta}{3} (1 - 3 c_s^2 v^2)^2  + \zeta  \right)\, .
\end{align}
Here $\gamma =1/\sqrt{1-v^2}$  and $c_s^2 = dp/de$ and we have neglected the bulk viscosity. 
The determinant of the matrix is positive definite  
\st
\noisekernel^{xxxx} \noisekernel^{\eta\eta\eta\eta} - (\noisekernel^{\eta\eta xx})^2 = \frac{1}{(1 - c_s^2 v^2)^2 \gamma^4}   \left(\frac{4}{3} \eta + \zeta \right)
\stp

% In Bjorken coordinates with only $x$ dependence we have
% the evolution equations from our paper:
% \begin{subequations}
%   \st
%   \partial_\tau (\tau \edense) + \partial_x (\tau \pdense^x) =  -  \left(p  + \tau^2\Pi^{\eta \eta} \right)  \,, \label{eq:equation} \\
%   \stp
%   and the momentum equation
%   \st
%   \partial_\tau (\tau \pdense^x) + \partial_x (\tau (\tideal^{xx} + \Pi^{xx})) = 0 \,.
%   \stp
% \end{subequations}
% The form of the RHS of the energy equation is a consequence of the geometry of Bjorken coordinates  and in a more general context could be written, 
% \st
% -  \left(p  + \tau^2\Pi^{\eta \eta} \right) = \sqrt{h} \left(\frac{K_{\eta\eta}}{\tau^2} \right)  \, \tau^2T^{\eta\eta} 
% \stp



Using the equation of motion it is easy to evaluate entropy production. 
The entropy density on a constant $\tau$ slice  is defined as in ideal hydrodynamics,  $S
= s(\beta) u^\tau$,  where the intermediate parameters $\beta_{\mu}(\edense,
\pdense^x)$ are functions of the conserved charges.  In
terms of these charges
\st
dS = -\beta_\tau d\edense  - \beta_x d\pdense^x \, .
\stp
Based on this equation we can evaluate the change in the entropy density by taking the time derivative of $S$ 
\st
 \partial_\tau (\tau S) = -\beta_{\tau} \partial_\tau (\tau \edense)  -\beta_x  \partial_\tau(\tau \pdense^x)\, , 
\stp
and upon using the equations of motion, we find
\st
\partial_\tau( \tau S)  + \partial_x (\tau S v^x -  \beta_x \, \tau \Pi^{xx}) =  \tau 
\begin{pmatrix}
  \partial_x \beta_x   &   -\beta_\tau \Gamma^{\tau}_{\eta\eta}
\end{pmatrix}
\begin{pmatrix}
  T\kappa^{xxxx}  &  T \kappa^{xx\eta\eta} \\
  T \kappa^{\eta\eta xx} & T  \kappa^{\eta\eta\eta\eta} 
\end{pmatrix}
\begin{pmatrix}
  \partial_x \beta_x   \\    -\beta_\tau \Gamma^{\tau}_{\eta\eta}   
\end{pmatrix} \, .
\stp
The right hand side is positive definite as long as the matrix of transport coefficients is positive definite. 
% The factor  $\beta^\tau/\tau$ in the entroy production reflects the extrinsic curvature 
% \begin{equation}
%   \frac{\beta^\tau}{\tau} = - \beta^\tau K_{\eta\eta}/\tau^2 \, .
% \end{equation}
% apearing in the covariant derivatives in \cref{eq:covderiv}.


We will define the dissipative function $R$ as half the positive definite quadratic form that appears in the entropy production:
\begin{equation}
R[T\kappa^{ijlm}, \beta_i, \beta^\tau] =  \frac{1}{2} \int  \tau  \, \dd x  \, \nabla_{i}\beta_{j} \, \left[  T\kappa^{ijmn}  \right] \,  \nabla_{m}\beta_{n}  
\end{equation}
Regarding the $T\kappa^{ijmn}$ as a constant metric on the space of strains $\nabla_i \beta_j$,  the viscous strains are given by the functional derivative of $R$ with respect to $\nabla_i \beta_j$:
\st
\Gamma^{\tau}_{\eta\eta} \Pi^{\eta\eta} =  \frac{1}{\tau} \frac{\delta R}{\delta
\beta_{\tau}(x)}   \qquad 
\partial_x (\tau \Pi^{xx}) =  \frac{1}{\tau} \frac{\delta R}{\delta \beta_x(x)} \, , \qquad
\stp
When evaluating the rate of entropy production, i.e.  
\begin{equation}
\frac{d\mathbb{S}}{d\tau} \equiv \frac{d}{d\tau} \int \tau \dd x\, S(\tau, x)\, , 
\end{equation}
upon using the equations of motion, the ideal terms do not contribute, and the viscous terms can be written in terms of the dissipative function as
\begin{equation}
 \frac{d\mathbb{S}}{d\tau} = \int \dd x \, \beta_\tau \,  (\tau \, \Gamma^{\tau}_{\eta\eta} \Pi^{\eta\eta}) + \beta_x \, \partial_x (\tau\Pi^{xx}) = \int \dd x  \left[  \beta_\tau(x) 
\frac{\delta R}{\delta
\beta_{\tau}(x)} +  \beta_x(x) \frac{\delta R}{\delta \beta_x(x)}  \right]
 = 2 R \, .
\end{equation}
When devising a numerical method,  we will write down the discrete version of the dissipative function and implement the discretized viscous strains by taking derivatives of this function. 
%  \begin{align}
%   \parital_x \Pi^{xx} 
%  \end{align}
 %=& - \frac{1}{\sqrt{h}} 
 % \frac{\delta R}{\delta \beta_x(x)} \\
  %K_{\eta\eta} \Pi^{\eta\eta} =& - \frac{1}{\sqrt{h}} \, \frac{\delta R}{\delta \beta^{\tau}(x)} 

% However,  $\delta \beta_{\mu}$ enters into the spatial stress, $T^{xx} =\tideal^{xx}(\beta+\delta\beta) + \underline{\Pi}^{xx}$. 
% Expanding $\tideal^{xx}(\beta + \delta \beta)$ to first order,  and using the ideal equations of motion
% % , e.g.  
% % \st
% %   \partial_{\mu} \beta^{\mu} \simeq  \frac{1}{\gamma^2}  \frac{1}{(1 - c_s^2 v^2)  }   \partial_x\beta_x
% % \stp
% to replace the time derivatives in \eqref{eq:Landaustress} with spatial derivatives  produces \eq{eq:txx1d} with $\kappa^{xxxx}(v)$ defined in \eqref{eq:noisekernelintro}.  These steps are detailed in \app{app:dfdetails}. 

% Next we will analyze entropy production using the Density Frame equations of motion. The entropy current in ideal hydrodynamics  is 
% \st
%  S^\mu= (S, S v^x) = s(\beta) u^\mu \, , 
% \stp
% where the intermediate parameters $\beta_{\mu}$ are defined by \eqref{eq:densityframedef}.  This definition of $S\equiv s(\beta) u^0$ (with $\beta_\mu$ defined by $\edense$ and $\pdense$) remains valid in the Density Frame, but the spatial current is modified.  Using the equations of motion it is easy to show  that
% \st
% \partial_t S + \partial_x (S v^x - \beta_x \Pi^{xx})  =  \kappa^{xxxx}(v) \left(\partial_x\beta_x\right)^2 \, .
% \stp
% Thus,  it is important that $\kappa^{xxxx}(v)$ in \eqref{eq:noisekernelintro} is a non-negative constant. 

% In our companion paper we present the equations of motion in  3+1D including the shear and bulk viscosities~\cite{TheoryPaper}.
% % \cmtJF{I know it is already said in the introduction, but should we remind people that a full 3+1D discussion is in the other paper?}


\subsection{Numerical method}
\label{sec:numerics}

As in our previous work we will use operator splitting together with an Implicit-Explicit (IMEX) time integrators~\cite{ASCHER1997151,Pareschi2005,Constantinescu} to
evolve the system.
The variables
are denoted generically\footnote{For clarity, we will drop the directional superscript in the $1+1$D case,  $\pdense \equiv \pdense^x$.  } $V \equiv(\edense, \pdense)$ and evolve schematically as
\st
  \dot V = f(V) + g(V)\,,
\stp
where $f(V)$ and $g(V)$ represent the ideal and viscous hydrodynamic evolution respectively.
The ideal evolution $f(V)$ is strongly non-linear and will be treated explicitly, while the viscous evolution $g(V)$ will be 
treated implicitly.

In the explicit step, $\dot V = f(V)$,  the variables are evolved over
a time interval $\Delta \tau$ using the equations of motion:
  \begin{subequations}
  \label{eq:FofV}
\begin{align}
  \partial_\tau \edense + \partial_x \pdense =& - \frac{\edense + p(\beta) }{\tau}\, ,   \\
    \partial_\tau \pdense + \partial_x \tideal^{xx}(\beta) =& - \pdense/\tau \, . 
\end{align}
  \end{subequations}
For the explicit step we use a standard Kurganov-Tadmor (KT) scheme of ideal hydrodynamics~\cite{KTScheme,KTAstro,Schenke}.

In the implicit step, $\dot V = g(V)$, the variables are evolved
with the viscous equations of motion,
\begin{subequations}
\label{eq:gofV}
\begin{align}
  \partial_\tau \edense
  &= - \Gamma^{\tau}_{\eta\eta} \Pi^{\eta\eta} \, , \\
  \partial_\tau \pdense
  &= - \partial_x \tvisc^{xx} \, .
\end{align}
\end{subequations}
At first order in $\Delta t$, the implicit update has the
schematic form
\st
\label{eq:gofVschematic}
  V^{n+1} = V^n + \Delta t \, g(V^{n+1}) \, ,
\stp
which in general is non-linear and must be
solved with a Newton solver. Combining the explicit and
implicit steps in sequence leads to a non-linear IMEX scheme.
Higher-order IMEX schemes are implemented in
PETSc~\cite{petsc-web-page}, which we use both for time stepping and
for solving the implicit non-linear equations, as in our previous
work~\cite{NumericsPaper}. Our remaining task is to specify in detail the
function $g(V)$ and its Jacobian
$J = \partial g / \partial V$ for the Newton solver.

To this end, we write down the discrete version of the dissipative function $R$. Then   
 taking derivatives of this function with respect to the variables at the $i$-th lattice site, $\beta_{x, i}$ and
$\beta_{\tau, i}$,  defines the discretization.  We will call the vertex located
halfway between the $i$ and $i{+}1$ lattice sites the $i+$ vertex, with an
analogous terminology for the $i-$ vertex. 
A generic variable $u$ at the $i+$ and $i-$ vertices is denoted  $u_{i+}$ and $u_{i-}$,  respectively.  These variables will be defined
as suitable averages  of the variables at the neighboring lattice sites (see
below). 
% which are defined as an average of the variables at the neighboring lattice sites: 
%   \st
%   u_{i+} \equiv  \frac{u_{i+1} + u_i}{2} \qquad 
%   u_{i-} \equiv \frac{u_{i} + u_{i-1}}{2} 
%   \stp
  We then introduce the \emph{discrete} dissipative function
  \st
  R(T\kappa, \beta) \equiv \tau\Delta x\sum_{i} R_{i+}  \, , 
  \stp 
  where $R_{i+}$ is the contribution to the dissipative function from the $i+$ vertex
  \begin{equation}
  R_{i+} \equiv \frac{1}{2}
\begin{pmatrix}
  \frac{\beta_{x, i+1} - \beta_{x,i}}{\Delta x}  &   -\Gamma^{\tau}_{\eta\eta} \frac{\beta_{\tau, i+1} + \beta_{\tau, i} }{2}
\end{pmatrix}
 \begin{pmatrix}
   T\kappa^{xxxx}   &  T \kappa^{xx\eta\eta} \\
   T\kappa^{\eta\eta xx}   &  T\kappa^{\eta\eta\eta\eta}
 \end{pmatrix}_{i+}
\begin{pmatrix}
  \frac{\beta_{x, i+1} - \beta_{x, i}}{\Delta x}  \\[0.5em]   -\Gamma^{\tau}_{\eta\eta} \frac{\beta_{\tau, i+1} + \beta_{\tau, i} }{2}
\end{pmatrix} \, .
\end{equation}
% We have introduce several abbreviations here to keep the notation compact,  $\kappa^{xx}\equiv T\kappa^{xxxx}$, $\kappa^{x\eta} \equiv T \kappa^{xx\eta\eta}$, and $\kappa^{\eta\eta} \equiv T \kappa^{\eta\eta\eta\eta}$, and $K = K_{\eta\eta}$. 
Then  the discritezed visous strains are given by
\st
(\Gamma^{\tau}_{\eta\eta} \tvisc^{\eta\eta})_i \equiv  
\frac{1}{\tau\Delta x}\frac{\partial R }{\partial \beta_{\tau, i}}  \, , 
\stp
and
\st
(\partial_x \tvisc^{xx})_i\equiv  \frac{1}{\tau\Delta x}\frac{\partial R }{\partial \beta_{x, i}} \, , 
\stp
respectively, which gives  concrete meaning to \cref{eq:gofV}. 
% The discretized viscous evolution equations are then given by
% \begin{align}
%    \partial_\tau \edense_i =&  - (\Gamma^{\tau}_{\eta\eta} \Pi^{\eta\eta})_i \,  \\
%    \partial_\tau \pdense_i =&  - (\partial_x \Pi^{xx})_i \, 
% \end{align}
As in the continuum, $R$ is a function $T\kappa$ and $\beta_{x}$ and
$\beta_{\tau}$ and the  derivatives of $R(T\kappa, \beta)$ with respect to $\beta_{x, i}$ and
$\beta_{\tau, i}$ are to be evaluated by treating $\kappa$ as a constant
metric on the space of strains.  In practice we evaluate $T\kappa^{ijlm}$ at $i+$ by 
averaging the conserved charges $(\edense, \pdense)$ of the neighboring lattice site,  using these
averages to define $\beta_{\mu, i+}$,  and finally evaluating
$T\kappa^{ijlm}(\beta_{i+})$.

There are several benefits to defining the discretization in this way.  First we
note that  the difference structure of the conservation equations is
automatically implemented by the definition of $R$. For example,  the viscous
contribution to the momentum equation is given by the difference of the viscous
stress at the neighboring lattice sites
\st
(\partial_x\Pi^{xx})_i = \frac{\partial R_{i+} }{\partial \beta_{x, i}} + \frac{\partial R_{i-} }{\partial \beta_{x, i}}  
 = 
  \frac{\partial R_{i+} }{\partial \beta_{x, i}} - \frac{\partial R_{i-} }{\partial \beta_{x, i-1}} \equiv  
 \frac{1}{\Delta x} \left(\Pi^{xx}_{i+} - \Pi^{xx}_{i-} \right) \, . 
\stp
%  while for the energy equation we have a simple average 
%  \st
%  (\Gamma^{\tau}_{\eta\eta} \Pi^{\eta\eta})_i =
%  \frac{\partial R_{i+} }{\partial \beta_{\tau, i}} + 
%  \frac{\partial R_{i-} }{\partial \beta_{\tau, i}}  
%  = 
%  \frac{\partial R_{i+} }{\partial \beta_{\tau, i}} + 
%  \frac{\partial R_{i-} }{\partial \beta_{\tau, i-1}} \equiv
%  \Gamma^{\tau}_{\eta\eta} \left(\Pi^{\eta\eta}_{i+} + \Pi^{\eta\eta}_{i-}\right)/2\, .
%  \stp

Further,  with this definition of $R$ entropy production is automatically positive definite  in a semi-discrete sense.  To see this,  we note that the total rate of entropy production on the lattice is given by
\begin{align}
  \frac{d\mathbb{S}}{d\tau} =&  \frac{d}{d\tau} \left( \sum_i \tau \Delta x S_i \right) =
\Delta x \sum_i -\beta_{\tau, i} \,  \partial_\tau (\tau\edense_i)   - \beta_{x, i} \,   \partial_\tau (\tau \pdense_i) 
\end{align}
which follows from the definition of the entropy.
Using the semi-discrete equations of motion,  the right hand side can be written as
\st
\frac{d\mathbb{S}}{d\tau} = 
%\Delta x \sum_i \left[  \beta^\tau_i \,  (\tau K_{\eta\eta} \Pi^{\eta\eta})_i + \beta_{x, i} \, \partial_x (\tau\Pi^{xx})_i \right] = 2 R
% \sum_i \beta^\tau  \partial_\tau (\tau\edense)   - \beta_{x} \,   \partial_\tau
% (\tau \pdense) = 
\tau \Delta x \sum_i \beta_{\tau, i}  (\Gamma^{\tau}_{\eta\eta} \Pi^{\eta\eta})_i + \beta_{x, i} \,
(\partial_x  \Pi^{xx})_i =   \sum_i \beta_{\tau, i} \frac{\partial R}{\partial \beta_{\tau, i}}
+ \beta_{x, i} \frac{\partial R}{\partial\beta_{x, i}} = 2 R  \, .
\stp
Thus the total rate of entropy production on the lattice is given by $2R$ as
in the continuum.

Now we will write down the explicit form of the viscous strains by differentiating $R$ and introducing a compact notation for the finite differences.
% We will write down the contribution of the $R_{i+}$ to 
% the $(\Gamma^{\tau}_{\eta\eta} \Pi^{\eta\eta})_i$ and $(\partial_x \Pi^{xx})_i$ a
% $(\Gamma^{\tau}_{\eta\eta} \Pi^{\eta\eta})_{i+1}$ and $(\partial_x \Pi^{xx})_{i+1}$. 
Let $\sigma=0,1$ denote the relative position with respect to the $i$-th lattice site. $\beta_x(\sigma)$  means $\beta_{x, i}$ for $\sigma = 0$ and $\beta_{x, i+1}$ for $\sigma = 1$.  We then introduce the following notation for the finite difference operators
\begin{equation}
  \Kd(\sigma) \equiv   \left(-\frac{1}{2} \Gamma^{\tau}_{\eta\eta}, -\frac{1}{2} \Gamma^{\tau}_{\eta\eta} \right) \, , 
   \qquad D_x(\sigma) \equiv  \left(-\frac{1}{\Delta x}, \frac{1}{\Delta x} \right)  \, , 
\end{equation}
for $\sigma = 0$ and $\sigma = 1$ respectively.
Then the form of $R_{i+}$ is given by 
  \begin{multline}
  R_{i+} = \frac{1}{2}  D_x(\sigma) \beta_x(\sigma) \, T\kappa^{xxxx} \, D_x(\sigma') \beta_x(\sigma') 
   +   \Kd(\sigma) \beta_\tau(\sigma) \, T\kappa^{\eta\eta xx} \, D_x(\sigma') \beta_x(\sigma')   \\
    + \frac{1}{2}  \Kd(\sigma) \beta_\tau(\sigma) \, T\kappa^{\eta\eta\eta\eta} \, \Kd(\sigma') \beta_\tau(\sigma') \, , 
  \end{multline}
where we are summing over the repeated indices $\sigma$ and $\sigma'$. Differentiating with respect to $\beta_{\tau}(\sigma)$ and $\beta_{x}(\sigma)$ yields the appropriate finite difference contribution to the viscous strains at the $i$-th and $(i+1)$-th lattice sites from $R_{i+}$. Thus, the discrete equation 
for the time evolution is  given in part by
\begin{subequations}
  \label{eq:gofVdetail}
\begin{align}
    -\partial_\tau \edense_{i+\sigma}  \supset \frac{\partial R_{i+}}{\partial \beta_\tau(\sigma)} =&  \,   
    \Kd(\sigma) \, T\kappa^{\eta\eta\eta\eta} \, \Kd(\sigma')\beta_\tau(\sigma') 
     + \Kd(\sigma) \, T\kappa^{\eta\eta xx} \, D_x(\sigma')\beta_x(\sigma')  \, , \\
   -\partial_\tau \pdense_{i+\sigma} \supset \frac{\partial R_{i+}}{\partial \beta_x(\sigma)} =& \,   
     D_x(\sigma) T\kappa^{xx \eta\eta} \Kd(\sigma')\beta_\tau(\sigma') + D_x(\sigma) T\kappa^{xxxx} D_{x} (\sigma') \beta_x(\sigma') \, .
\end{align}
\end{subequations}
The complete time evolution is found by summing over the contributions from all  vertices
(i.e. $i+$ and $i-$ for lattice site $i$) to yield 
$(\partial_x \Pi^{xx})_i$ in its complete form. To summarize,
\cref{eq:gofVdetail} defines the function $g(V)$ in the implicit equation
$\partial_\tau V = g(V)$.

The procedure is now easily extended to the 2+1D case.  The two dimensional 
lattice sites are labeled by the vector $i=(i_x, i_y)$  
and the relative postions are labelled by the vector $\sigma = (\sigma_x, \sigma_y)$ where $\sigma_x$ and $\sigma_y$ can take the values $0$ and $1$.  
$\beta_x(\sigma)$ means $\beta_{x, i+\sigma}$ for 
$\sigma \in \{ \sigma_0, \sigma_1, \sigma_2, \sigma_3\}$ where $\sigma_0 = (0,0)$, $\sigma_1 = (1,0)$, $\sigma_2 = (0,1)$, and $\sigma_3 = (1,1)$.  
We will write down the contribution of $R_{i+(1/2, 1/2)}$ to the viscous strains. 
 The finite difference operators are defined analogously to the 1+1D case,
\begin{subequations}
\begin{align}
  D_x(\sigma)\beta_a(\sigma) \equiv & \frac{1}{2\Delta x} (-\beta_{a, i+\sigma_0} + \beta_{a, i+\sigma_1} - \beta_{a, i+\sigma_2} + \beta_{a, i+\sigma_3}) \, ,  \\
  D_y(\sigma)\beta_a(\sigma) \equiv & \frac{1}{2\Delta y} (-\beta_{a, i+\sigma_0} - \beta_{a, i+\sigma_1} + \beta_{a, i+\sigma_2} + \beta_{a, i+\sigma_3}) \, , \\ 
   K^\tau(\sigma) \beta_\tau(\sigma) \equiv & -\Gamma^\tau_{\eta\eta} \frac{1}{4} (\beta_{\tau, i+\sigma_0} + \beta_{\tau, i+\sigma_1} + \beta_{\tau, i+\sigma_2} + \beta_{\tau, i+\sigma_3}) \, .
\end{align}
\end{subequations}
% \begin{multline}
%   R_{i+(1/2, 1/2)} = \frac{1}{2}  D_a(\sigma) \beta_b(\sigma') \, \kappa^{abcd} \, D_c(\sigma') \beta_d(\sigma) 
%    +   K(\sigma) \beta_\tau(\sigma) \, \kappa^{\eta\eta ab} \, D_a(\sigma') \beta_b(\sigma')   \\
%     + \frac{1}{2}  K(\sigma) \beta_\tau(\sigma) \, \kappa^{\eta\eta\eta\eta} \, K(\sigma') \beta_\tau(\sigma')
% \end{multline}
The equations of motion are decorated with transverse indices 
\begin{subequations}
\label{eq:gofVdetail2d}
\begin{align}
     -\partial_\tau \edense_{i+\sigma}  \supset \frac{\partial R_{i+(1/2, 1/2)}}{\partial \beta_\tau(\sigma)} =&  \,   
    \Kd(\sigma) \, T\kappa^{\eta\eta\eta\eta} \, \Kd(\sigma')\beta_\tau(\sigma') 
     + \Kd(\sigma) \, T\kappa^{\eta\eta ab} \, D_a(\sigma')\beta_b(\sigma')  \, ,  \\
   -\partial_\tau \pdense_{i+\sigma}^a \supset \frac{\partial R_{i+(1/2, 1/2)}}{\partial \beta_a(\sigma)} =& \,   
     D_b(\sigma) T\kappa^{a b \eta\eta} \Kd(\sigma')\beta_\tau(\sigma') + D_b(\sigma) T\kappa^{abcd} D_{c} (\sigma') \beta_d(\sigma') \, .
\end{align}
\end{subequations}

To write down an approximate Jacobian for implicit solver, we have to evaluate the
derivatives of the rhs \cref{eq:gofVdetail2d} with respect to 
to the variables on the lattice $V_i \equiv T^{\tau \mu}_i \equiv (\edense_i, \pdense_i)$. 
In keeping with the spirit of the dissipative function, we will treat
$T\kappa$ as
a constant, when taking these derivatives.
The Jacobian of the discretized
operator then invloves the 
thermodynamic susceptibility  defined by
\begin{align}
\chi^{-1}_{\mu\nu} =&  \frac{\partial \beta_{\mu}}{\partial T^{\tau \nu}}  \, , 
\end{align}
 Explicit formulas for this susceptibility matrix given in our previous work~\cite{NumericsPaper,TheoryPaper}.
 Using a compact notation  \cref{eq:gofVdetail}  can be written as
 % and its derivative with respect to $T^{\tau\mu}$ is given 
\begin{align}
 \partial_\tau T^{\tau \mu}_{i+\sigma}  \supset& \;  g^{\mu\rho}_{\sigma\sigma'}\, \beta_{\rho}(\sigma')  \, , 
\end{align}
and thus the  contribution of $R_{i+(1/2, 1/2)}$ to the Jacobian at sites $i+\sigma$ and $i+ \sigma'$ is 
\begin{align}
 (J_{\sigma, \sigma'})^{\mu}_{\nu} = g^{\mu\rho}_{\sigma\sigma'} \, \chi^{-1}_{\rho\nu}(\sigma')  \qquad (\mbox{no sum over $\sigma'$}). 
\end{align}
Adding the contributions of this sort from all of the vertices,  determines the full Jacobian  matrix for the Newton Solver, $J(V) = \partial g/\partial V$.  
%or more explicity   $(J_{i_1,i_2})^{\mu}_{\nu} = \partial g_{i_1}^{\mu}/\partial V_{i_2}^{\nu}$.




So far we  have presented a fully non-linear IMEX scheme. A simpler
alternative to a fully non-linear implicit step, known as Extrapolated IMEX (EIMEX), is to linearize
the problem over a time interval $\Delta t$~\cite{Constantinescu}.
We adopted this approach in our previous numerical work on the
Density Frame~\cite{NumericsPaper} and found it to be as
effective and robust as the fully non-linear method, while being
being easier to implement and faster too.

In this approach, we solve the linearized equation
\st
\partial_t \, \delta V(t)
= f(V^n) + g(V^n) + J(V^n)\,\delta V \,,
\stp
where $\delta V(t) = V(t) - V^n$ and
$J(V^n) = \partial g/\partial V$ is a time-independent
approximate Jacobian matrix written down in the preceding paragraphs.
This leads to the update
\st
V^{n+1}
= V^n
+ \frac{\Delta t \, (f(V^n) + g(V^n))}
       {1 - \Delta t J(V^n)} \,,
\stp
so that the increment $\delta V$ is obtained by solving the
linear system
\st
(1 - \Delta t J(V^n)) \, \delta V
= \Delta t \, (f(V^n) + g(V^n)) \,.
\stp
The first order EIMEX scheme presented here,  as well as its higher-order generalizations, are all implemented in
PETSc~\cite{petsc-web-page}.  We adopted a third-order EIMEX scheme studied in
our previous work~\cite{NumericsPaper}.
We used PETSc both for EIMEX time
stepping and to solve the linear systems arising in the
linearized treatment. In practice, the default linear solver in
PETSc---a GMRES solver~\cite{trefethen1997numerical} with an
ILU preconditioner---has proven to be efficient for these
linear systems.



\section{Numerical tests in 2+1 dimensions}

\subsection{Inclined shock tube test}

In our previous work we presented the results of a viscous shock tube test in
1+1D and compared the results to simulations of QCD kinetic theory and to MUSIC.
For the 2+1D case, we will perform the same test, but with the shock tube
inclined at an angle to the lattice axes.  

\subsection{Comparison to Gubser solution}

Some references to the Gubser solution~\cite{Gubser:2010ze,Gubser:2010ui} and ~\cite{Denicol:2014tha, Dash:2020zqx}

\subsection{Comparison to MUSIC for typical heavy ion initial conditions}

\section{Conclusions and Outlook}
\bibliographystyle{apsrev4-2}
\bibliographystyle{utphys}
\bibliography{refs2d}
\end{document}
% We write the index $A$ as $\sigma\mu$ where $\sigma$ loops over the lattice cites
% \st
% U_{\sigma \mu } = S^t_{\sigma \mu I}  \kappa^{\eta\eta \eta\eta} 
% \stp

% \st
% S_{IA} \beta_A = D_{\mu_I}(\sigma) \beta_{\nu_I}(\sigma)
% \stp
% We take 
% \st
%  \chi^{-1}_{AC}  = \chi^{-1}_{\mu\mu'}(\sigma) \delta_{\sigma \sigma'}
% \stp
% Then 
% \st
% S_{IA } \chi^{-1}_{AC} = D_{\mu_I}(\sigma) \chi^{-1}_{\mu_I \mu'}(\sigma)
% \stp
% Then 
% \st
% S^t_{\sigma \mu J} \kappa^{JI} = D_{\nu}(\sigma) \kappa^{J_{\mu\nu} I}  
% \stp


%\bibliographystyle{apsrev4-2}
%\bibliographystyle{utphys}
%\bibliography{pv_ref}
%\begin{thebibliography}{99} 
%\input{numerics2.bbl}
%\end{thebibliography}
\end{document}
